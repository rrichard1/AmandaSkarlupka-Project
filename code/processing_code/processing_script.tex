\documentclass[]{article}
\usepackage{lmodern}
\usepackage{amssymb,amsmath}
\usepackage{ifxetex,ifluatex}
\usepackage{fixltx2e} % provides \textsubscript
\ifnum 0\ifxetex 1\fi\ifluatex 1\fi=0 % if pdftex
  \usepackage[T1]{fontenc}
  \usepackage[utf8]{inputenc}
\else % if luatex or xelatex
  \ifxetex
    \usepackage{mathspec}
  \else
    \usepackage{fontspec}
  \fi
  \defaultfontfeatures{Ligatures=TeX,Scale=MatchLowercase}
\fi
% use upquote if available, for straight quotes in verbatim environments
\IfFileExists{upquote.sty}{\usepackage{upquote}}{}
% use microtype if available
\IfFileExists{microtype.sty}{%
\usepackage{microtype}
\UseMicrotypeSet[protrusion]{basicmath} % disable protrusion for tt fonts
}{}
\usepackage[margin=1in]{geometry}
\usepackage{hyperref}
\hypersetup{unicode=true,
            pdftitle={processing\_script},
            pdfauthor={Amanda Skarlupka},
            pdfborder={0 0 0},
            breaklinks=true}
\urlstyle{same}  % don't use monospace font for urls
\usepackage{color}
\usepackage{fancyvrb}
\newcommand{\VerbBar}{|}
\newcommand{\VERB}{\Verb[commandchars=\\\{\}]}
\DefineVerbatimEnvironment{Highlighting}{Verbatim}{commandchars=\\\{\}}
% Add ',fontsize=\small' for more characters per line
\usepackage{framed}
\definecolor{shadecolor}{RGB}{248,248,248}
\newenvironment{Shaded}{\begin{snugshade}}{\end{snugshade}}
\newcommand{\AlertTok}[1]{\textcolor[rgb]{0.94,0.16,0.16}{#1}}
\newcommand{\AnnotationTok}[1]{\textcolor[rgb]{0.56,0.35,0.01}{\textbf{\textit{#1}}}}
\newcommand{\AttributeTok}[1]{\textcolor[rgb]{0.77,0.63,0.00}{#1}}
\newcommand{\BaseNTok}[1]{\textcolor[rgb]{0.00,0.00,0.81}{#1}}
\newcommand{\BuiltInTok}[1]{#1}
\newcommand{\CharTok}[1]{\textcolor[rgb]{0.31,0.60,0.02}{#1}}
\newcommand{\CommentTok}[1]{\textcolor[rgb]{0.56,0.35,0.01}{\textit{#1}}}
\newcommand{\CommentVarTok}[1]{\textcolor[rgb]{0.56,0.35,0.01}{\textbf{\textit{#1}}}}
\newcommand{\ConstantTok}[1]{\textcolor[rgb]{0.00,0.00,0.00}{#1}}
\newcommand{\ControlFlowTok}[1]{\textcolor[rgb]{0.13,0.29,0.53}{\textbf{#1}}}
\newcommand{\DataTypeTok}[1]{\textcolor[rgb]{0.13,0.29,0.53}{#1}}
\newcommand{\DecValTok}[1]{\textcolor[rgb]{0.00,0.00,0.81}{#1}}
\newcommand{\DocumentationTok}[1]{\textcolor[rgb]{0.56,0.35,0.01}{\textbf{\textit{#1}}}}
\newcommand{\ErrorTok}[1]{\textcolor[rgb]{0.64,0.00,0.00}{\textbf{#1}}}
\newcommand{\ExtensionTok}[1]{#1}
\newcommand{\FloatTok}[1]{\textcolor[rgb]{0.00,0.00,0.81}{#1}}
\newcommand{\FunctionTok}[1]{\textcolor[rgb]{0.00,0.00,0.00}{#1}}
\newcommand{\ImportTok}[1]{#1}
\newcommand{\InformationTok}[1]{\textcolor[rgb]{0.56,0.35,0.01}{\textbf{\textit{#1}}}}
\newcommand{\KeywordTok}[1]{\textcolor[rgb]{0.13,0.29,0.53}{\textbf{#1}}}
\newcommand{\NormalTok}[1]{#1}
\newcommand{\OperatorTok}[1]{\textcolor[rgb]{0.81,0.36,0.00}{\textbf{#1}}}
\newcommand{\OtherTok}[1]{\textcolor[rgb]{0.56,0.35,0.01}{#1}}
\newcommand{\PreprocessorTok}[1]{\textcolor[rgb]{0.56,0.35,0.01}{\textit{#1}}}
\newcommand{\RegionMarkerTok}[1]{#1}
\newcommand{\SpecialCharTok}[1]{\textcolor[rgb]{0.00,0.00,0.00}{#1}}
\newcommand{\SpecialStringTok}[1]{\textcolor[rgb]{0.31,0.60,0.02}{#1}}
\newcommand{\StringTok}[1]{\textcolor[rgb]{0.31,0.60,0.02}{#1}}
\newcommand{\VariableTok}[1]{\textcolor[rgb]{0.00,0.00,0.00}{#1}}
\newcommand{\VerbatimStringTok}[1]{\textcolor[rgb]{0.31,0.60,0.02}{#1}}
\newcommand{\WarningTok}[1]{\textcolor[rgb]{0.56,0.35,0.01}{\textbf{\textit{#1}}}}
\usepackage{graphicx,grffile}
\makeatletter
\def\maxwidth{\ifdim\Gin@nat@width>\linewidth\linewidth\else\Gin@nat@width\fi}
\def\maxheight{\ifdim\Gin@nat@height>\textheight\textheight\else\Gin@nat@height\fi}
\makeatother
% Scale images if necessary, so that they will not overflow the page
% margins by default, and it is still possible to overwrite the defaults
% using explicit options in \includegraphics[width, height, ...]{}
\setkeys{Gin}{width=\maxwidth,height=\maxheight,keepaspectratio}
\IfFileExists{parskip.sty}{%
\usepackage{parskip}
}{% else
\setlength{\parindent}{0pt}
\setlength{\parskip}{6pt plus 2pt minus 1pt}
}
\setlength{\emergencystretch}{3em}  % prevent overfull lines
\providecommand{\tightlist}{%
  \setlength{\itemsep}{0pt}\setlength{\parskip}{0pt}}
\setcounter{secnumdepth}{0}
% Redefines (sub)paragraphs to behave more like sections
\ifx\paragraph\undefined\else
\let\oldparagraph\paragraph
\renewcommand{\paragraph}[1]{\oldparagraph{#1}\mbox{}}
\fi
\ifx\subparagraph\undefined\else
\let\oldsubparagraph\subparagraph
\renewcommand{\subparagraph}[1]{\oldsubparagraph{#1}\mbox{}}
\fi

%%% Use protect on footnotes to avoid problems with footnotes in titles
\let\rmarkdownfootnote\footnote%
\def\footnote{\protect\rmarkdownfootnote}

%%% Change title format to be more compact
\usepackage{titling}

% Create subtitle command for use in maketitle
\providecommand{\subtitle}[1]{
  \posttitle{
    \begin{center}\large#1\end{center}
    }
}

\setlength{\droptitle}{-2em}

  \title{processing\_script}
    \pretitle{\vspace{\droptitle}\centering\huge}
  \posttitle{\par}
    \author{Amanda Skarlupka}
    \preauthor{\centering\large\emph}
  \postauthor{\par}
      \predate{\centering\large\emph}
  \postdate{\par}
    \date{10/26/2019}


\begin{document}
\maketitle

load the packages needed

\begin{Shaded}
\begin{Highlighting}[]
\KeywordTok{library}\NormalTok{(readxl)}
\KeywordTok{library}\NormalTok{(knitr)}
\KeywordTok{library}\NormalTok{(tidyverse)}
\end{Highlighting}
\end{Shaded}

\begin{verbatim}
## -- Attaching packages --------------------------------------------------------------------------------------- tidyverse 1.2.1 --
\end{verbatim}

\begin{verbatim}
## v ggplot2 3.2.1          v purrr   0.3.2     
## v tibble  2.1.3          v dplyr   0.8.3     
## v tidyr   1.0.0.9000     v stringr 1.4.0     
## v readr   1.3.1          v forcats 0.4.0
\end{verbatim}

\begin{verbatim}
## -- Conflicts ------------------------------------------------------------------------------------------ tidyverse_conflicts() --
## x dplyr::filter() masks stats::filter()
## x dplyr::lag()    masks stats::lag()
\end{verbatim}

\begin{Shaded}
\begin{Highlighting}[]
\KeywordTok{library}\NormalTok{(readxl)}
\KeywordTok{library}\NormalTok{(gtools)}
\KeywordTok{library}\NormalTok{(gdata)}
\end{Highlighting}
\end{Shaded}

\begin{verbatim}
## gdata: read.xls support for 'XLS' (Excel 97-2004) files ENABLED.
\end{verbatim}

\begin{verbatim}
## 
\end{verbatim}

\begin{verbatim}
## gdata: read.xls support for 'XLSX' (Excel 2007+) files ENABLED.
\end{verbatim}

\begin{verbatim}
## 
## Attaching package: 'gdata'
\end{verbatim}

\begin{verbatim}
## The following objects are masked from 'package:dplyr':
## 
##     combine, first, last
\end{verbatim}

\begin{verbatim}
## The following object is masked from 'package:purrr':
## 
##     keep
\end{verbatim}

\begin{verbatim}
## The following object is masked from 'package:stats':
## 
##     nobs
\end{verbatim}

\begin{verbatim}
## The following object is masked from 'package:utils':
## 
##     object.size
\end{verbatim}

\begin{verbatim}
## The following object is masked from 'package:base':
## 
##     startsWith
\end{verbatim}

load the data. The data is contained within one excel file, but is
separated on different sheets

\begin{Shaded}
\begin{Highlighting}[]
\NormalTok{p1_mab_full <-}\StringTok{ }\KeywordTok{read_excel}\NormalTok{(}\StringTok{"~/Documents/Data Science/AmandaSkarlupka-Project/data/raw_data/data.xlsx"}\NormalTok{, }
                      \DataTypeTok{sheet =} \StringTok{"p1_mab"}\NormalTok{, }\DataTypeTok{na =} \StringTok{"NA"}\NormalTok{)}
\NormalTok{ca09_mab_full <-}\StringTok{ }\KeywordTok{read_excel}\NormalTok{(}\StringTok{"~/Documents/Data Science/AmandaSkarlupka-Project/data/raw_data/data.xlsx"}\NormalTok{, }
                      \DataTypeTok{sheet =} \StringTok{"ca09_mab"}\NormalTok{, }\DataTypeTok{na =} \StringTok{"NA"}\NormalTok{)}
\NormalTok{p1_sera_full <-}\StringTok{ }\KeywordTok{read_excel}\NormalTok{(}\StringTok{"~/Documents/Data Science/AmandaSkarlupka-Project/data/raw_data/data.xlsx"}\NormalTok{, }
                        \DataTypeTok{sheet =} \StringTok{"p1_sera"}\NormalTok{, }\DataTypeTok{na =} \StringTok{"NA"}\NormalTok{)}
\NormalTok{ca09_sera_full <-}\StringTok{ }\KeywordTok{read_excel}\NormalTok{(}\StringTok{"~/Documents/Data Science/AmandaSkarlupka-Project/data/raw_data/data.xlsx"}\NormalTok{, }
                        \DataTypeTok{sheet =} \StringTok{"ca09_sera"}\NormalTok{, }\DataTypeTok{na =} \StringTok{"NA"}\NormalTok{)}
\NormalTok{antigen_key <-}\StringTok{ }\KeywordTok{read_excel}\NormalTok{(}\StringTok{"~/Documents/Data Science/AmandaSkarlupka-Project/data/raw_data/data.xlsx"}\NormalTok{, }
                   \DataTypeTok{sheet =} \StringTok{"antigen_key"}\NormalTok{, }\DataTypeTok{na =} \StringTok{"NA"}\NormalTok{)}
\NormalTok{antibody_key <-}\StringTok{ }\KeywordTok{read_excel}\NormalTok{(}\StringTok{"~/Documents/Data Science/AmandaSkarlupka-Project/data/raw_data/data.xlsx"}\NormalTok{,}
                           \DataTypeTok{sheet =} \StringTok{"antibody_key"}\NormalTok{, }\DataTypeTok{na =} \StringTok{"NA"}\NormalTok{)}
\end{Highlighting}
\end{Shaded}

\#Take a look at the data and make sure that it loaded correctly

\begin{Shaded}
\begin{Highlighting}[]
\KeywordTok{summary}\NormalTok{(p1_mab_full)}
\end{Highlighting}
\end{Shaded}

\begin{verbatim}
##   clone_name          Spain/2003       Zhejiang/07         Swine/31    
##  Length:12          Min.   : 0.0780   Min.   : 0.0780   Min.   : 1.25  
##  Class :character   1st Qu.: 0.2145   1st Qu.: 0.9765   1st Qu.:20.00  
##  Mode  :character   Median : 0.9375   Median : 1.8750   Median :20.00  
##                     Mean   : 7.5260   Mean   : 8.8834   Mean   :16.88  
##                     3rd Qu.:20.0000   3rd Qu.:20.0000   3rd Qu.:20.00  
##                     Max.   :20.0000   Max.   :20.0000   Max.   :20.00  
##   Illinois/09  Minnesota/09      Nebraska/13       Iowa/73       
##  Min.   :20   Min.   : 0.1562   Min.   : 1.25   Min.   : 0.1562  
##  1st Qu.:20   1st Qu.: 0.5469   1st Qu.: 8.75   1st Qu.:17.5000  
##  Median :20   Median :15.0000   Median :20.00   Median :20.0000  
##  Mean   :20   Mean   :11.0547   Mean   :15.10   Mean   :16.4714  
##  3rd Qu.:20   3rd Qu.:20.0000   3rd Qu.:20.00   3rd Qu.:20.0000  
##  Max.   :20   Max.   :20.0000   Max.   :20.00   Max.   :20.0000  
##      WI/97          Colorado/09       NC/34543/09          MN/15      
##  Min.   : 0.3125   Min.   : 0.1560   Min.   : 0.1560   Min.   : 2.50  
##  1st Qu.: 1.2500   1st Qu.: 0.5859   1st Qu.: 0.5859   1st Qu.:20.00  
##  Median :10.0000   Median : 2.5000   Median : 3.7500   Median :20.00  
##  Mean   :10.4427   Mean   : 8.1510   Mean   : 6.6927   Mean   :17.29  
##  3rd Qu.:20.0000   3rd Qu.:20.0000   3rd Qu.:10.0000   3rd Qu.:20.00  
##  Max.   :20.0000   Max.   :20.0000   Max.   :20.0000   Max.   :20.00  
##     Utah/09       NC/09         Missouri/13     NC/15      Indiana/00    
##  Min.   :20   Min.   : 0.078   Min.   :20   Min.   :20   Min.   : 0.234  
##  1st Qu.:20   1st Qu.: 1.062   1st Qu.:20   1st Qu.:20   1st Qu.:20.000  
##  Median :20   Median : 6.250   Median :20   Median :20   Median :20.000  
##  Mean   :20   Mean   : 9.762   Mean   :20   Mean   :20   Mean   :18.353  
##  3rd Qu.:20   3rd Qu.:20.000   3rd Qu.:20   3rd Qu.:20   3rd Qu.:20.000  
##  Max.   :20   Max.   :20.000   Max.   :20   Max.   :20   Max.   :20.000  
##      NC/01     NC/5043-1/09   
##  Min.   :20   Min.   : 0.156  
##  1st Qu.:20   1st Qu.: 2.188  
##  Median :20   Median :20.000  
##  Mean   :20   Mean   :12.213  
##  3rd Qu.:20   3rd Qu.:20.000  
##  Max.   :20   Max.   :20.000
\end{verbatim}

\begin{Shaded}
\begin{Highlighting}[]
\KeywordTok{str}\NormalTok{(p1_mab_full)}
\end{Highlighting}
\end{Shaded}

\begin{verbatim}
## Classes 'tbl_df', 'tbl' and 'data.frame':    12 obs. of  19 variables:
##  $ clone_name  : chr  "1F8" "3H6" "3D3" "2A5" ...
##  $ Spain/2003  : num  7.5 1.25 0.234 0.078 0.156 ...
##  $ Zhejiang/07 : num  20 1.25 2.5 0.078 0.156 20 20 0.117 1.25 1.25 ...
##  $ Swine/31    : num  20 20 20 1.25 20 1.25 20 20 20 20 ...
##  $ Illinois/09 : num  20 20 20 20 20 20 20 20 20 20 ...
##  $ Minnesota/09: num  20 0.312 20 1.25 0.312 ...
##  $ Nebraska/13 : num  20 20 5 1.25 5 20 20 10 20 20 ...
##  $ Iowa/73     : num  20 0.156 20 20 20 ...
##  $ WI/97       : num  20 1.875 10 1.25 0.625 ...
##  $ Colorado/09 : num  20 0.625 2.5 1.25 0.469 ...
##  $ NC/34543/09 : num  1.25 0.469 5 2.5 0.312 ...
##  $ MN/15       : num  20 5 20 20 20 20 20 20 20 2.5 ...
##  $ Utah/09     : num  20 20 20 20 20 20 20 20 20 20 ...
##  $ NC/09       : num  0.5 20 1.25 0.078 0.312 ...
##  $ Missouri/13 : num  20 20 20 20 20 20 20 20 20 20 ...
##  $ NC/15       : num  20 20 20 20 20 20 20 20 20 20 ...
##  $ Indiana/00  : num  0.234 20 20 20 20 20 20 20 20 20 ...
##  $ NC/01       : num  20 20 20 20 20 20 20 20 20 20 ...
##  $ NC/5043-1/09: num  20 2.5 20 20 1.25 20 2.5 0.156 20 0.156 ...
\end{verbatim}

\begin{Shaded}
\begin{Highlighting}[]
\KeywordTok{str}\NormalTok{(ca09_mab_full)}
\end{Highlighting}
\end{Shaded}

\begin{verbatim}
## Classes 'tbl_df', 'tbl' and 'data.frame':    18 obs. of  19 variables:
##  $ clone_name  : chr  "1E6" "2A12" "5B_2A12" "2B11" ...
##  $ Spain/2003  : num  0.156 20 0.039 20 20 20 10 2.5 20 0.039 ...
##  $ Zhejiang/07 : num  0.156 20 0.078 20 20 20 20 5 20 0.039 ...
##  $ Swine/31    : num  5 20 20 20 20 2.5 20 0.625 20 20 ...
##  $ Illinois/09 : num  20 20 20 20 20 20 20 20 20 20 ...
##  $ Minnesota/09: num  0.312 0.078 20 20 20 ...
##  $ Nebraska/13 : num  0.625 20 20 20 20 5 20 2.5 20 5 ...
##  $ Iowa/73     : num  20 20 20 20 20 20 1.25 5 20 0.156 ...
##  $ WI/97       : num  0.625 20 0.625 20 20 ...
##  $ Colorado/09 : num  0.469 20 0.156 10 20 5 1.25 1.25 20 0.234 ...
##  $ NC/34543/09 : num  0.312 5 0.156 5 20 ...
##  $ MN/15       : num  20 20 0.156 20 20 ...
##  $ Utah/09     : num  20 10 20 20 20 20 20 20 20 20 ...
##  $ NC/09       : num  20 20 20 20 20 20 20 20 20 20 ...
##  $ Missouri/13 : num  20 20 20 20 20 20 20 20 20 20 ...
##  $ NC/15       : num  20 20 20 20 20 20 20 20 20 20 ...
##  $ Indiana/00  : num  5 20 20 20 20 10 20 0.078 20 20 ...
##  $ NC/01       : num  20 20 20 20 20 20 20 20 20 20 ...
##  $ NC/5043-1/09: num  0.078 20 0.02 20 20 ...
\end{verbatim}

\begin{Shaded}
\begin{Highlighting}[]
\KeywordTok{str}\NormalTok{(p1_sera_full)}
\end{Highlighting}
\end{Shaded}

\begin{verbatim}
## Classes 'tbl_df', 'tbl' and 'data.frame':    10 obs. of  19 variables:
##  $ mouse       : num  1 2 3 4 5 6 7 8 9 10
##  $ Spain/2003  : num  320 NA 160 320 640 320 320 640 960 NA
##  $ Zhejiang/07 : num  320 NA 80 160 320 160 160 640 320 NA
##  $ Swine/31    : logi  NA NA NA NA NA NA ...
##  $ Illinois/09 : logi  NA NA NA NA NA NA ...
##  $ Minnesota/09: logi  NA NA NA NA NA NA ...
##  $ Nebraska/13 : logi  NA NA NA NA NA NA ...
##  $ Iowa/73     : num  320 480 5 640 240 320 5 640 640 160
##  $ WI/97       : num  240 320 160 320 480 240 80 160 400 80
##  $ Colorado/09 : num  960 960 160 320 960 480 320 960 480 320
##  $ NC/34543/09 : num  480 960 160 320 960 320 320 1280 480 320
##  $ MN/15       : num  5 80 5 5 5 5 5 5 5 5
##  $ Utah/09     : logi  NA NA NA NA NA NA ...
##  $ NC/09       : logi  NA NA NA NA NA NA ...
##  $ Missouri/13 : num  5 NA 5 5 5 5 5 5 5 NA
##  $ NC/15       : logi  NA NA NA NA NA NA ...
##  $ Indiana/00  : num  5 5 5 5 5 5 5 5 5 5
##  $ NC/01       : num  5 5 5 5 5 5 5 5 5 5
##  $ NC/5043-1/09: num  1280 2560 5 320 1280 5 640 2560 5 640
\end{verbatim}

\begin{Shaded}
\begin{Highlighting}[]
\KeywordTok{str}\NormalTok{(ca09_sera_full)}
\end{Highlighting}
\end{Shaded}

\begin{verbatim}
## Classes 'tbl_df', 'tbl' and 'data.frame':    6 obs. of  19 variables:
##  $ mouse       : num  1 2 3 4 5 6
##  $ Spain/2003  : num  480 40 240 80 80 160
##  $ Zhejiang/07 : num  320 5 160 5 5 160
##  $ Swine/31    : logi  NA NA NA NA NA NA
##  $ Illinois/09 : logi  NA NA NA NA NA NA
##  $ Minnesota/09: logi  NA NA NA NA NA NA
##  $ Nebraska/13 : logi  NA NA NA NA NA NA
##  $ Iowa/73     : num  5 5 5 5 5 5
##  $ WI/97       : num  160 40 240 80 80 160
##  $ Colorado/09 : num  640 80 960 160 120 320
##  $ NC/34543/09 : num  1920 320 1920 640 320 320
##  $ MN/15       : num  5 5 5 5 5 5
##  $ Utah/09     : logi  NA NA NA NA NA NA
##  $ NC/09       : logi  NA NA NA NA NA NA
##  $ Missouri/13 : num  5 5 5 5 5 5
##  $ NC/15       : logi  NA NA NA NA NA NA
##  $ Indiana/00  : num  5 5 5 5 5 5
##  $ NC/01       : num  5 5 5 5 5 5
##  $ NC/5043-1/09: num  2560 640 1280 1280 320 1280
\end{verbatim}

Move the monoclonal antibody concentrations and the sera titers into the
same column

\begin{Shaded}
\begin{Highlighting}[]
\NormalTok{p1_mab <-}\StringTok{ }\NormalTok{p1_mab_full }\OperatorTok
\StringTok{  }\KeywordTok{gather}\NormalTok{(}\DataTypeTok{key =} \StringTok{"antigen"}\NormalTok{, }\DataTypeTok{value =} \StringTok{"concentration"}\NormalTok{, }\DecValTok{2}\OperatorTok{:}\DecValTok{19}\NormalTok{)}

\NormalTok{ca09_mab <-}\StringTok{ }\NormalTok{ca09_mab_full }\OperatorTok
\StringTok{  }\KeywordTok{gather}\NormalTok{(}\DataTypeTok{key =} \StringTok{"antigen"}\NormalTok{, }\DataTypeTok{value =} \StringTok{"concentration"}\NormalTok{, }\DecValTok{2}\OperatorTok{:}\DecValTok{19}\NormalTok{)}

\NormalTok{p1_sera <-}\StringTok{ }\NormalTok{p1_sera_full }\OperatorTok
\StringTok{  }\KeywordTok{gather}\NormalTok{(}\DataTypeTok{key =} \StringTok{"antigen"}\NormalTok{, }\DataTypeTok{value =} \StringTok{"titer"}\NormalTok{, }\DecValTok{2}\OperatorTok{:}\DecValTok{19}\NormalTok{)}

\NormalTok{ca09_sera <-}\StringTok{ }\NormalTok{ca09_sera_full }\OperatorTok
\StringTok{  }\KeywordTok{gather}\NormalTok{(}\DataTypeTok{key =} \StringTok{"antigen"}\NormalTok{, }\DataTypeTok{value =} \StringTok{"titer"}\NormalTok{, }\DecValTok{2}\OperatorTok{:}\DecValTok{19}\NormalTok{)}
\end{Highlighting}
\end{Shaded}

The sera was not ran against all the viruses due to a limited supply. So
the viruses that were used for sera testing are filtered below.

\begin{Shaded}
\begin{Highlighting}[]
\NormalTok{ca09_sera <-}\StringTok{ }\NormalTok{ca09_sera }\OperatorTok
\StringTok{  }\KeywordTok{na.omit}\NormalTok{()}
\NormalTok{p1_sera <-}\StringTok{ }\NormalTok{p1_sera }\OperatorTok
\StringTok{  }\KeywordTok{na.omit}\NormalTok{()}
\end{Highlighting}
\end{Shaded}

The antibody concentrations are in two fold dilutions, the titers will
be log2(value)

\begin{Shaded}
\begin{Highlighting}[]
\NormalTok{p1_mab}\OperatorTok{$}\NormalTok{log2 <-}\StringTok{ }\KeywordTok{log2}\NormalTok{(p1_mab}\OperatorTok{$}\NormalTok{concentration)}
\NormalTok{p1_mab}\OperatorTok{$}\NormalTok{dilution <-}\StringTok{ }\KeywordTok{log2}\NormalTok{(}\DecValTok{1}\OperatorTok{/}\NormalTok{(p1_mab}\OperatorTok{$}\NormalTok{concentration}\OperatorTok{/}\DecValTok{20}\NormalTok{))}\OperatorTok{*}\DecValTok{2}

\NormalTok{p1_sera}\OperatorTok{$}\NormalTok{log2 <-}\StringTok{ }\KeywordTok{log2}\NormalTok{(p1_sera}\OperatorTok{$}\NormalTok{titer)}


\NormalTok{ca09_mab}\OperatorTok{$}\NormalTok{log2 <-}\StringTok{ }\KeywordTok{log2}\NormalTok{(ca09_mab}\OperatorTok{$}\NormalTok{concentration)}
\NormalTok{ca09_mab}\OperatorTok{$}\NormalTok{dilution <-}\StringTok{ }\KeywordTok{log2}\NormalTok{(}\DecValTok{1}\OperatorTok{/}\NormalTok{(ca09_mab}\OperatorTok{$}\NormalTok{concentration}\OperatorTok{/}\DecValTok{20}\NormalTok{))}\OperatorTok{*}\DecValTok{2}

\NormalTok{ca09_sera}\OperatorTok{$}\NormalTok{log2 <-}\StringTok{ }\KeywordTok{log2}\NormalTok{(ca09_sera}\OperatorTok{$}\NormalTok{titer)}
\end{Highlighting}
\end{Shaded}

\#The lineages of the viruses need to be added. The antigen\_key
contains all the information about the viruses.

\begin{Shaded}
\begin{Highlighting}[]
\CommentTok{#add the antigen key to the sera data for switch out the antigen with the antigen key for ease of script writing}

\NormalTok{ca09_sera <-}\StringTok{ }\NormalTok{ca09_sera }\OperatorTok
\StringTok{  }\KeywordTok{left_join}\NormalTok{(antigen_key, }\DataTypeTok{by =} \KeywordTok{c}\NormalTok{(}\StringTok{"antigen"}\NormalTok{ =}\StringTok{ "short_name"}\NormalTok{))}

\NormalTok{p1_sera <-}\StringTok{ }\NormalTok{p1_sera }\OperatorTok
\StringTok{  }\KeywordTok{left_join}\NormalTok{(antigen_key, }\DataTypeTok{by =} \KeywordTok{c}\NormalTok{(}\StringTok{"antigen"}\NormalTok{ =}\StringTok{ "short_name"}\NormalTok{))}

\NormalTok{ca09_mab <-}\StringTok{ }\NormalTok{ca09_mab }\OperatorTok
\StringTok{  }\KeywordTok{left_join}\NormalTok{(antibody_key, }\DataTypeTok{by =} \StringTok{"clone_name"}\NormalTok{)}

\NormalTok{p1_mab <-}\StringTok{ }\NormalTok{p1_mab }\OperatorTok
\StringTok{  }\KeywordTok{left_join}\NormalTok{(antibody_key, }\DataTypeTok{by =} \StringTok{"clone_name"}\NormalTok{)}
\end{Highlighting}
\end{Shaded}

save data as RDS

\begin{Shaded}
\begin{Highlighting}[]
\KeywordTok{saveRDS}\NormalTok{(ca09_mab, }\DataTypeTok{file =} \StringTok{"~/Documents/Data Science/AmandaSkarlupka-Project/data/processed_data/ca09mab_processed_data.rds"}\NormalTok{)}
\KeywordTok{saveRDS}\NormalTok{(ca09_sera, }\DataTypeTok{file =} \StringTok{"~/Documents/Data Science/AmandaSkarlupka-Project/data/processed_data/ca09sera_processed_data.rds"}\NormalTok{)}
\KeywordTok{saveRDS}\NormalTok{(p1_mab, }\DataTypeTok{file =} \StringTok{"~/Documents/Data Science/AmandaSkarlupka-Project/data/processed_data/p1mab_processed_data.rds"}\NormalTok{)}
\KeywordTok{saveRDS}\NormalTok{(p1_sera, }\DataTypeTok{file =} \StringTok{"~/Documents/Data Science/AmandaSkarlupka-Project/data/processed_data/p1sera_processed_data.rds"}\NormalTok{)}
\CommentTok{#saveRDS(p1_sera_full, file = "./data/processed_data/p1sera_full_processed_data.rds")}
\CommentTok{#saveRDS(CA09_sera_full, file = "./data/processed_data/CA09sera_full_processed_data.rds")}
\CommentTok{#saveRDS(p1_mabs_full, file = "./data/processed_data/p1mabs_full_processed_data.rds")}
\CommentTok{#saveRDS(CA09_mabs_full, file = "./data/processed_data/CA09mabs_full_processed_data.rds")}
\end{Highlighting}
\end{Shaded}


\end{document}
